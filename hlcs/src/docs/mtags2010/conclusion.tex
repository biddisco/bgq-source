% begin_generated_IBM_copyright_prolog                             %
%                                                                  %
% This is an automatically generated copyright prolog.             %
% After initializing,  DO NOT MODIFY OR MOVE                       %
% ================================================================ %
%                                                                  %
% (C) Copyright IBM Corp.  2011, 2011                              %
% Eclipse Public License (EPL)                                     %
%                                                                  %
% ================================================================ %
%                                                                  %
% end_generated_IBM_copyright_prolog                               %
\section{Conclusion}

Many-Task Computing describes an emerging application style for large scale computing. 
It cuts across both HTC and HPC application paradigms but has sufficient enough 
characteristics to warrant its own classification as a computing paradigm. In this
paper we have described how Blue Gene architecture has transitioned from the original
BG/L machine that
specialized in large scale HPC workloads to the latest BG/Q system that will be able to tackle 
a multitude of 
customer's MTC workloads under a unified 
Control System software model. BG/Q is just a way station on the journey to exaflop 
supercomputing though. New challenges and workloads await and the Blue Gene architecture 
must continue to evolve to meet the future requirements of exaflop computing. 
With a sustained record of success in supercomputing, all indications point 
to the fact that the elegant and flexible architecture of Blue Gene is prepared 
to meet those challenges. 

