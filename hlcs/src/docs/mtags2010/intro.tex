% begin_generated_IBM_copyright_prolog                             %
%                                                                  %
% This is an automatically generated copyright prolog.             %
% After initializing,  DO NOT MODIFY OR MOVE                       %
% ================================================================ %
%                                                                  %
% (C) Copyright IBM Corp.  2011, 2011                              %
% Eclipse Public License (EPL)                                     %
%                                                                  %
% ================================================================ %
%                                                                  %
% end_generated_IBM_copyright_prolog                               %
\section{Introduction}
Blue Gene/Q (BG/Q) is the third generation computer architecture in the Blue Gene family of supercomputers. 
The BG/Q system will be capable of scaling to over a million processor cores while making the trade-off of 
lower power consumption over raw processor speed.

Blue Gene systems are connected to multiple communications networks. BG/L, the first generation member of the
Blue Gene supercomputers, and BG/P, the second generation, both provide a three dimensional (3D) torus network that is used for peer-to-peer
communication between compute nodes. BG/Q advances the technology by supporting a five dimensional (5D) torus network. All the Blue Gene 
systems incorporate a collective network for collective communication operations and a global interrupt network for fast barriers.
An Ethernet network provides communication to external I/O attached to Blue Gene while a private Ethernet network
provides access for managing hardware resources.   

By default a custom lightweight operating system called Compute Node Kernel (CNK) is loaded on compute nodes while I/O nodes
run the Linux operating system. I/O nodes were integrated on the same board as compute nodes for BG/L and BG/P. The BG/Q
hardware design moves the I/O nodes to separate I/O drawers and I/O racks.

Just as the Blue Gene hardware has advanced over multiple generations, the resource management software on Blue Gene 
known as the Control System has also evolved to support the latest supercomputer workload models.

This position paper describes the BG/Q resource management software design at an early stage of development.
While the software architecture is in place, the development team is not able to share performance results
with the community at this point in time.
